\documentclass[10pt,a4paper,sans]{moderncv}

\moderncvstyle{classic}
% CV theme - options include: 'casual', 'classic', 'oldstyle', 'banking'
\moderncvcolor{blue}
% CV color - options include: 'blue' (default), 'orange', 'green', 'red',
% 'purple', 'grey' and 'black'

\usepackage[scale=0.9]{geometry} % Reduce document margins
%\setlength{\hintscolumnwidth}{3cm}
%\setlength{\makecvtitlenamewidth}{10cm} % For the 'classic' style.

%------------------------------------------------
%	NAME AND CONTACT INFORMATION SECTION
%------------------------------------------------

\firstname{Revanth} % Your first name
\familyname{Bhattaram} % Your last name

% All information in this block is optional.
%\title{Grad Student, Carnegie Mellon University}
\mobile{(412) 944-9750}
\email{rbhattar@andrew.cmu.edu, revanthb3000@gmail.com}
\homepage{http://www.github.com/revanthb3000/}

%------------------------------------------------

\begin{document}

\makecvtitle % Print the CV title

%------------------------------------------------
%	EDUCATION SECTION
%------------------------------------------------

\section{Education}

\cventry{Dec. 2015}{Masters in Computer Science (CQPA : 4.0/4.0)}{Carnegie
Mellon, School of Computer Science}{PA}{}{
\begin{itemize}
\item TA for 15-451 - Algorithm Design and Analysis (Spring '15).
\item TA for 10-601 - Introduction to Machine Learning (Fall '15).
\end{itemize}
}
\cventry{July 2014}{Bachelors in Computer Science (CGPA : 9.07/10)}{Indian
Institute of Technology}{Guwahati, India}{}{}

%------------------------------------------------
%	WORK EXPERIENCE SECTION
%------------------------------------------------

\section{Experience}

\cventry{Summer '15}{Software Engineering Intern (Ads Integrity Infra)}{
Facebook}{Menlo Park, CA}{}
{
\begin{itemize}
\item Developed a tool that extracts \textbf{image, text and landing page}
features by traversing the DOM structure of an ad preview. The tool has gains
over the existing system for 38\% of all ad impressions.
\item Worked on the development of a \textbf{web crawler} to help automate the
DOM Parser pipeline.
\item System is designed to be \textbf{scalable} and is run through every ad
that goes into Facebook (over a million ads per day).
\end{itemize}}

\cventry{Summer '13}{Software Engineering Intern}{Amazon Development Centre}{
Hyderabad, India}{}
{
\begin{itemize}
\item Developed \textbf{Integration Testing Frameworks} for internal team
services that automatically run integration tests each time a change is pushed.
\item Used Selenium Webdriver to develop a similar framework for the team's
internal website.
\end{itemize}}

%------------------------------------------------

\cventry{Summer '12}{Intern}{International Institute Of Information Technology}{
Hyderabad, India}{}
{
\begin{itemize}
\item Worked on the development of \textbf{eAgromet} (eagromet.in), an
agro-meteorological advisory system being developed by IIIT Hyderabad, ANGRAU
and the IMD.
\item Work included the development of a user hierarchy, a user management
module, a weather module, advice and bulletin generation module along with a
search engine which would search for similar weather patterns.
\end{itemize}}

%-----------------------------------------------

%-----------------------------------------------
%	PROJECTS SECTION
%-----------------------------------------------

\section{Projects}
\cventry{Feb.2015--May.2015}{\textbf{Robot Restaurant}}{}{}{}
{
\begin{itemize}
\item The setting is a restaurant that is run entirely by automated robots. The
goal is to come up with a way for the robots to learn their roles. This is
viewed as a multi agent reinforcement learning problem.
\item The world is modeled as an MDP and the 'learning' part is handled using
Q-Learning with various reward schemes and learning strategies.
\item The problem of scalability is handled by using the technique of
co-ordination graphs.
\end{itemize}}

\cventry{Sep.2014--Dec.2014}{\textbf{American Epilepsy Society Seizure
Prediction Challenge}}{}{}{}
{
\begin{itemize}
\item This was part of a kaggle competition where the task was to predict the
occurrence seizures given iEEG readings from different regions of a patient's
brain.
\item Characterized data generated by electrodes using Fourier analysis.
Observed correlation between electrodes and their relation with the occurrence
of seizures.
\item Applied various Machine Learning algorithms such as Naive Bayes,
Regression Trees, SVM, Kernel SVM and Boosting.
\end{itemize}}

\cventry{Sep.2014--Dec.2014}{\textbf{Graph Miner}}{}{}{}
{
\begin{itemize}
\item Surveyed and implemented various graphmining algorithms such as Pagerank,
K-core decomposition etc..
\item Observed global patterns and anomalies in several real world datasets
(from various fields like social networking, online retail, Internet traffic
etc..) using the developed package.
\end{itemize}}

\cventry{Apr.2013--May.2014}{\textbf{Recommendation System on Mobile Phones
(Bachelor Thesis Project)}}{}{}{}
{
\begin{itemize}
\item Project aims to build a recommendation system based off a user's mobile
activities. Diversity of information obtained from a user's device is what sets
it apart from traditional recommendation systems.
\item Phase 1 was spent in building a system to classify a user's activities. A
 Naive Bayes text classifier with features extracted using Chi Square/Gini
 coefficients was used.
\item Phase 2 was spent in building a user network where collaborative filtering
 with weighted similarity scores was used to provide recommendations.
\end{itemize}}

\cventry{Dec.2012}{\textbf{RCPrep.com}}{}{}{}
{
\begin{itemize}
\item RCPrep is a reading comprehension test simulator to help aspirants in
crack the Reading Comprehensions section of various exams.
\item The website has received over 300,000 page requests till now and has been
on an upward trend since launch.
\item The website was developed using Web2Py and an MVC architecture.
\end{itemize}}


\section{Computer Skills}

\cvitem{Languages}{JAVA, Python, PHP, Hack, JavaScript, Matlab, SQL, C++, C,
Bash}
\cvitem{Technologies}{Hadoop, Hive, Spark, Presto, Selenium, HHVM, Django,
Web2Py, React, FlowType}

\end{document}
